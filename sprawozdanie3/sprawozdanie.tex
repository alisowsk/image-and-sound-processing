% vim:encoding=utf8 ft=tex sts=2 sw=2 et:

\documentclass{classrep}
\usepackage[utf8]{inputenc}
\usepackage{color}
\usepackage{mathtools}

\usepackage{subfig}
\usepackage{float}

\usepackage[labelfont=it]{caption}

\studycycle{Informatyka, studia niestacjonarne, mgr II st.}
\coursesemester{I}

\coursename{Przetwarzanie obrazu i dźwięku}
\courseyear{2015/2016}

\courseteacher{mgr inż. Piotr Ożdżyński}
\coursegroup{Sobota, 14:15}

\author{
  \studentinfo{Jakub Antosik}{206788} \and
  \studentinfo{Andrzej Lisowski}{206807} 
}

\title{Zadanie 3: Analiza częstotliwości podstawowej dźwięku.}

\begin{document}
\maketitle

\section{Cel}
Celem zadania było zapoznanie się z metodami analizy dźwięku, a w szczególności znajdowania okresu i częstotliwości. Badane były dwie grupy metod: realizowane w dziedzinie czasu oraz w dziedzinie częstotliwości. W części implementacyjnej należało stworzyć program w wybranym przez siebie języku programowania, który będzie w stanie przeprowadzić po jednej, wybranej operacji operacji z z każdej z grup. Do tego celu wykorzystano szkielet apliakcji z zadań 1 i 2.\\
\indent
Szczegółowy opis zadania został przedstawiony w [1].

\section{Wprowadzenie}
//TODO

\subsection{Autokorelacja}
Autokorelacja jest korelacją sygnału z samym sobą, w kolejnych punktach w czasie. Innymi słowy, autokorelacja informuje o tym na ile dane wartości / obserwacje są istotnie związane z obserwacjami zaobserwowanymi wcześniej (o stałym przesunięciu czasowym) [3].\\
\indent
Wzór na autokorelację sygnału dyskretnego jest następujący:\\
\\
\[ c(m) = \sum_{n = 1}^{N - 1} x(n) \ast x(n + m) \]
\\

\subsection{Analiza widma Fouriera sygnału}
//TODO

\section{Opis implementacji}
Opis implementacji został przedstawiony w sprawozdaniu do zadania 1 [2]. Zadanie 3 zostało zrealizowane poprzez rozszerzenie funkcjonalności programu. Dodany został nowy interfejs graficzny, dedykowany dla przetwarzania dźwięku oraz analizowane metody tj. autokorelację oraz analizę widma Fouriera sygnału.

\section{Materiały i metody}
Do aplikacji dodano szereg testowych dźwięków w celu dokładnej analizy badanych metod. Ich spis zamieszczono poniżej:\\
\indent
\begin{itemize}
\item Sztuczne
\begin{itemize}
\item Łatwe: 100Hz, 150Hz, 225Hz, 337Hz, 506Hz, 759Hz, 1139Hz, 1708Hz
\item Średnie: 90Hz, 135Hz, 202Hz, 303Hz, 455Hz, 683Hz, 1025Hz, 1537Hz
\item Trudne: 80Hz, 120Hz, 180Hz, 270Hz, 405Hz, 607Hz, 911Hz, 1366Hz
\end{itemize}
\item Naturalne
\begin{itemize}
\item Flet: 276Hz, 443Hz, 591Hz, 887Hz, 1265Hz, 1779Hz
\item Altówka: 130Hz, 196Hz, 247Hz, 294Hz, 369Hz, 440Hz, 698Hz
\end{itemize}
\item Sekwencje
\begin{itemize}
\item DWK altówka
\item KDF pianino
\end{itemize}
\end{itemize}

\section{Wyniki}
Sekcja prezentuje wyniki przeprowadzanego badania metody autokorelacji i analizy widma Fouriera sygnału. 

\subsection{Autokorelacja}
W poniższej tabeli przedstawiono częstotliwości badanych dźwięków - faktyczną oraz znalezioną w wyniku autokorelacji.\\
\\
\begin{tabular}{ l | c | c }
  \hline
  Dźwięk testowy & Autokorelacja & Analiza widma Fouriera \\
  \hline			
  Sztuczne, łatwe, 100Hz & 100 & TODO \\
  Sztuczne, łatwe, 150Hz & 150 & TODO \\
  Sztuczne, łatwe, 225Hz & 225 & TODO \\
  Sztuczne, łatwe, 337Hz & 336 & TODO \\
  Sztuczne, łatwe, 506Hz & 506 & TODO \\
  Sztuczne, łatwe, 759Hz & 760 & TODO \\
  Sztuczne, łatwe, 1139Hz & 1130 & TODO \\
  Sztuczne, łatwe, 1708Hz & 1696 & TODO \\
  \hline
  Sztuczne, średnie, 90Hz & 90 & TODO \\
  Sztuczne, średnie, 135Hz & 135 & TODO \\
  Sztuczne, średnie, 202Hz & 202 & TODO \\
  Sztuczne, średnie, 303Hz & 302 & TODO \\
  Sztuczne, średnie, 455Hz & 454 & TODO \\
  Sztuczne, średnie, 683Hz & 678 & TODO \\
  Sztuczne, średnie, 1025Hz & 1025 & TODO \\
  Sztuczne, średnie, 1537Hz & 1520 & TODO \\
  \hline 
  Sztuczne, trudne, 80Hz & 80 & TODO \\
  Sztuczne, trudne, 120Hz & 120 & TODO \\
  Sztuczne, trudne, 180Hz & 180 & TODO \\
  Sztuczne, trudne, 270Hz & 270 & TODO \\
  Sztuczne, trudne, 405Hz & 404 & TODO \\
  Sztuczne, trudne, 607Hz & 604 & TODO \\
  Sztuczne, trudne, 911Hz & 454 & TODO \\
  Sztuczne, trudne, 1366Hz & 341 & TODO \\
  \hline 
  Naturalne, flet, 276Hz & 649 & TODO \\
  Naturalne, flet, 443Hz & 2913 & TODO \\
  Naturalne, flet, 591Hz & 2618 & TODO \\
  Naturalne, flet, 887Hz & 2078 & TODO \\
  Naturalne, flet, 1265Hz & 2941 & TODO \\
  Naturalne, flet, 1779Hz & 4148 & TODO \\
  \hline 
  Naturalne, altówka, 130Hz & 1738 & TODO \\
  Naturalne, altówka, 196Hz & 4741 & TODO \\
  Naturalne, altówka, 247Hz & 5404 & TODO \\
  Naturalne, altówka, 294Hz & 4406 & TODO \\
  Naturalne, altówka, 369Hz & 1853 & TODO \\
  Naturalne, altówka, 440Hz & 4426 & TODO \\
  Naturalne, altówka, 698Hz & 3678 & TODO \\
  \hline 
\end{tabular}

\subsection{Analiza widma Fouriera sygnału}
//TODO

\section{Dyskusja i wnioski}
Poniższa sekcja prezentuje interpretację uzyskanych wyników oraz wnioski. Opisano również napotkane problemy oraz możliwe sposoby ich rozwiązania.

\subsection{Autokorelacja}
Metoda autokorelacji rozpoznała częstotliwość dźwięków wygenerowanych sztucznie z bardzo dużą dokładnością. Miała jedynie problemy z trudnymi dźwiękami, o najwyższej częstotliwości. Niestety, analizowana metoda nie potrafiła rozpoznać ani jednej częstoliwości w dźwiękac naturalnych, we flecie i w altówce. Wynika to najprawdopodobniej z szumu otoczenia, który powinien być usunięty przed rozpoczęciem autokorelacji.

\subsection{Analiza widma Fouriera sygnału}
//TODO

\begin{thebibliography}{1}
\bibitem{instruction_pol}\text{$http://ftims.edu.p.lodz.pl/pluginfile.php/20101/mod\_resource/content/1/$}\\
\text{$Third2012.pdf, 2015$}
\bibitem{instruction_pol}\text{$https://github.com/alisowsk/image-and-sound-processing/blob/master/$}\\
\text{$sprawozdanie/sprawozdanie.pdf, 2015$}
\bibitem{instruction_pol}\text{$http://www.naukowiec.org/wiedza/statystyka/autokorelacja\_410.html, 2015$}\\
\end{thebibliography}

\end{document}
