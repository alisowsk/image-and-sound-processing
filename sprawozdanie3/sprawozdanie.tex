% vim:encoding=utf8 ft=tex sts=2 sw=2 et:

\documentclass{classrep}
\usepackage[utf8]{inputenc}
\usepackage{color}
\usepackage{mathtools}

\usepackage{subfig}
\usepackage{float}

\usepackage[labelfont=it]{caption}

\studycycle{Informatyka, studia niestacjonarne, mgr II st.}
\coursesemester{I}

\coursename{Przetwarzanie obrazu i dźwięku}
\courseyear{2015/2016}

\courseteacher{mgr inż. Piotr Ożdżyński}
\coursegroup{Sobota, 14:15}

\author{
  \studentinfo{Jakub Antosik}{206788} \and
  \studentinfo{Andrzej Lisowski}{206807} 
}

\title{Zadanie 3: Analiza częstotliwości podstawowej dźwięku.}

\begin{document}
\maketitle

\section{Cel}
Celem zadania było zapoznanie się z metodami analizy dźwięku, a w szczególności znajdowania okresu i częstotliwości. Badane były dwie grupy metod: realizowane w dziedzinie czasu oraz w dziedzinie częstotliwości. W części implementacyjnej należało stworzyć program w wybranym przez siebie języku programowania, który będzie w stanie przeprowadzić po jednej, wybranej operacji operacji z z każdej z grup. Do tego celu wykorzystano szkielet apliakcji z zadań 1 i 2.\\
\indent
Szczegółowy opis zadania został przedstawiony w [1].

\section{Wprowadzenie}
Jedną z najważniejszych cech dźwięku jest jego częstotliwość. Człowiek potrafi usłyszeć częstotliwości zawarte są w paśmie pomiędzy wartościami granicznymi od ok. 16 Hz do ok. 20 kHz. Istnieje wiele sposobów na jej rozpoznanie. W poniższej pracy przeanalizowano dwa: przy pomocy autokorelacji i metody cepstralnej.
 
\subsection{Autokorelacja}
Autokorelacja jest korelacją sygnału z samym sobą, w kolejnych punktach w czasie. Innymi słowy, autokorelacja informuje o tym na ile dane wartości / obserwacje są istotnie związane z obserwacjami zaobserwowanymi wcześniej (o stałym przesunięciu czasowym) [3].\\
\indent
Wzór na autokorelację sygnału dyskretnego jest następujący:\\
\\
\[ c(m) = \sum_{n = 1}^{N - 1} x(n) \ast x(n + m) \]
\\

\subsection{Analiza cepstralna}
Analiza cepstralna jest metodą wykrywania częstotliwości dźwięku przy pomocy transformaty Fouriera. W celu usunięcia szumów, na próbkę należy nałożyć okno. Częstotliwość podstawowa jest wykrywana na podstawie rozkładu maximów w dziedzinie cepstrum.  

\section{Opis implementacji}
Opis implementacji został przedstawiony w sprawozdaniu do zadania 1 [2]. Zadanie 3 zostało zrealizowane poprzez rozszerzenie funkcjonalności programu. Dodany został nowy interfejs graficzny, dedykowany dla przetwarzania dźwięku oraz analizowane metody tj. autokorelację oraz analizę cepstalną.\\
\indent 
Wykrywanie częstotliwości sekwencji zostało przeprowadzone poprzez podział dźwięku na N równych części i analizę każdej próbki osobno.

\section{Materiały i metody}
Do aplikacji dodano szereg testowych dźwięków w celu dokładnej analizy badanych metod. Ich spis zamieszczono poniżej:\\
\indent
\begin{itemize}
\item Sztuczne
\begin{itemize}
\item Łatwe: 100Hz, 150Hz, 225Hz, 337Hz, 506Hz, 759Hz, 1139Hz, 1708Hz
\item Średnie: 90Hz, 135Hz, 202Hz, 303Hz, 455Hz, 683Hz, 1025Hz, 1537Hz
\item Trudne: 80Hz, 120Hz, 180Hz, 270Hz, 405Hz, 607Hz, 911Hz, 1366Hz
\end{itemize}
\item Naturalne
\begin{itemize}
\item Flet: 276Hz, 443Hz, 591Hz, 887Hz, 1265Hz, 1779Hz
\item Altówka: 130Hz, 196Hz, 247Hz, 294Hz, 369Hz, 440Hz, 698Hz
\end{itemize}
\item Sekwencje
\begin{itemize}
\item DWK altówka
\item KDF pianino
\end{itemize}
\end{itemize}

\section{Wyniki}
Sekcja prezentuje wyniki przeprowadzanego badania metody autokorelacji i analizy widma Fouriera sygnału.\\ 
\\
W poniższej tabeli przedstawiono częstotliwości badanych dźwięków - faktyczną oraz znalezioną w wyniku autokorelacji oraz analizy cepstralnej.\\
\\
\begin{tabular}{ l | c | c }
  \hline
  Dźwięk testowy & Autokorelacja & Analiza cepstralna \\
  \hline			
  Sztuczne, łatwe, 100Hz & 100 & 100 \\
  Sztuczne, łatwe, 150Hz & 150 & 150 \\
  Sztuczne, łatwe, 225Hz & 226 & 441 \\
  Sztuczne, łatwe, 337Hz & 339 & 336 \\
  Sztuczne, łatwe, 506Hz & 512 & 506 \\
  Sztuczne, łatwe, 759Hz & 773 & 760 \\
  Sztuczne, łatwe, 1139Hz & 1160 & 1160 \\
  Sztuczne, łatwe, 1708Hz & 1764 & 1696 \\
  \hline
  Sztuczne, średnie, 90Hz & 90 & 94 \\
  Sztuczne, średnie, 135Hz & 135 & 129 \\
  Sztuczne, średnie, 202Hz & 203 & 367 \\
  Sztuczne, średnie, 303Hz & 304 & 302 \\
  Sztuczne, średnie, 455Hz & 459 & 454 \\
  Sztuczne, średnie, 683Hz & 689 & 689 \\
  Sztuczne, średnie, 1025Hz & 1050 & 1025 \\
  Sztuczne, średnie, 1537Hz & 1575 & 1520 \\
  \hline 
  Sztuczne, trudne, 80Hz & 80 & 32 \\
  Sztuczne, trudne, 120Hz & 120 & 80 \\
  Sztuczne, trudne, 180Hz & 180 & 90 \\
  Sztuczne, trudne, 270Hz & 272 & 38 \\
  Sztuczne, trudne, 405Hz & 408 & 270 \\
  Sztuczne, trudne, 607Hz & 612 & 604 \\
  Sztuczne, trudne, 911Hz & 938 & 364 \\
  Sztuczne, trudne, 1366Hz & 1422 & 2756 \\
  \hline 
  Naturalne, flet, 276Hz & 277 & 273 \\
  Naturalne, flet, 443Hz & 445 & 441 \\
  Naturalne, flet, 591Hz & 595 & 595 \\
  Naturalne, flet, 887Hz & 900 & 445 \\
  Naturalne, flet, 1265Hz & 1297 & 1260 \\
  Naturalne, flet, 1779Hz & 1837 & 1696 \\
  \hline 
  Naturalne, altówka, 130Hz & 130 & 130 \\
  Naturalne, altówka, 196Hz & 196 & 196 \\
  Naturalne, altówka, 247Hz & 247 & 246 \\
  Naturalne, altówka, 294Hz & 295 & 294 \\
  Naturalne, altówka, 369Hz & 370 & 367 \\
  Naturalne, altówka, 440Hz & 445 & 441 \\
  Naturalne, altówka, 698Hz & 700 & 700 \\
  \hline 
  Rozpoznano & 22/24 szt. 12/13 nat. & 15/24 szt. 12/13 nat. \\
  \hline 
\end{tabular}

\section{Dyskusja i wnioski}
Poniższa sekcja prezentuje interpretację uzyskanych wyników oraz wnioski. Opisano również napotkane problemy oraz możliwe sposoby ich rozwiązania.

\subsection{Autokorelacja - pojedyncze częstotliwości}
Metoda autokorelacji rozpoznała częstotliwość dźwięków wygenerowanych sztucznie z bardzo dużą dokładnością. Można zauważyć, że im wyższe częstotliwości, tym większe były rozbieżoności. Analizowana metoda potrafiła rozpoznać również częstotliwości naturalne, we flecie i w altówce.

\subsection{Autokorelacja - sekwencje}
Autokorelacja poprawnie rozpoznała częstotliwości w sekwencji altówki i pianina. Najlepsze wyniki uzyskano dla wielkości próbki: 4000(altówka) i 10000(pianino). Odmienne wartości są spowodowane dłuższymi dźwiękami pianina. Niestety, zapis do pliku wynikowego dodaje szum pomiędzy kolejnymi częstotliwościami.

\subsection{Metoda cepstralna}
Analiza cepstralna również dobrze rozpoznała większość częstotliwości. Miała jednak duży problem z wykryciem częstotliwości trudnych dźwięków sztucznych - odniosła sukces tylko w jednym przypadku. 

\subsection{Metoda cepstralna - sekwencje}
Analiza metodą cepstralną dała zadowalające wyniki wygenerowania sekwencji dźwięków altówki dla próbek o wielkości 8192. Słychać jednak wyraźnie fragmenty rozpoznane błędnie. Niestety, nie udało się wygenerować poprawnej sekwencji pianina - najlepsze wyniki uzyskano dla prbek o wielkości 16384 i 32768.

\begin{thebibliography}{1}
\bibitem{instruction_pol}\text{$http://ftims.edu.p.lodz.pl/pluginfile.php/20101/mod\_resource/content/1/$}\\
\text{$Third2012.pdf, 2015$}
\bibitem{instruction_pol}\text{$https://github.com/alisowsk/image-and-sound-processing/blob/master/$}\\
\text{$sprawozdanie/sprawozdanie.pdf, 2015$}
\bibitem{instruction_pol}\text{$http://www.naukowiec.org/wiedza/statystyka/autokorelacja\_410.html, 2015$}\\
\end{thebibliography}

\end{document}
