% vim:encoding=utf8 ft=tex sts=2 sw=2 et:

\documentclass{classrep}
\usepackage[utf8]{inputenc}
\usepackage{color}
\usepackage{mathtools}

\studycycle{Informatyka, studia niestacjonarne, mgr II st.}
\coursesemester{I}

\coursename{Przetwarzanie obrazu i dźwięku}
\courseyear{2015/2016}

\courseteacher{mgr inż. Piotr Ożdżyński}
\coursegroup{Sobota, 14:15}

\author{
  \studentinfo{Jakub Antosik}{XXXXXX} \and
  \studentinfo{Andrzej Lisowski}{206087} 
}

\title{Zadanie 1: Szkielet aplikacji do przetwarzania i analizy obrazów, operacje podstawowe, usuwanie szumu,
modyfikacje histogramu, filtracja liniowa i nieliniowa, splot.}

\begin{document}
\maketitle

\section{Cel}
Celem zadania było zapoznanie się z metodami analizy i przetwarzania obrazów. W części implementacyjnej należało stworzyć program w wybranym przez siebie języku programowania, który będzie w stanie przeprowadzić różne operacje na obrazie. Pełen spis funkcjonalności zostanie przedstawiony w sekcji \textit{Wprowadzenie}.

\section{Wprowadzenie}
Obraz w pamięci komputera jest reprezentowany przez macierz pikseli. Sam piksel jest zaś najmniejszym elementem obrazu, mogącym przyjmować różne wartości liczb naturalnych:\\
\begin{itemize}
\item 0 - 7 dla obrazów 1-bitowych\\
\item 0 - 255 dla obrazów 8-bitowych (odcienie szarości)\\
\item 0 - 16777215 dla obrazów 24-bitowych (po 8 bitów na każdy kolor RGB)\\
\end{itemize}
Poniżej przedstawione zostały teoretyczne podstawy trasformacji, którym poddane zostały testowe dane.

\subsection{Podstawowe operacje przetwarzania obrazu}
Sekcja ta opisuje podstawowe operacje przetwarzania obrazów, często wykorzystywane w codziennym życiu.

\subsubsection{Zmiana jasności}
Zmiana jasności obrazu polega na dodaniu do wartości każdego piksela pewnej stałej liczby k. Jeżeli stała jest dodatnia, mówimy o zwiększaniu jasności, jeżeli jest ujemna - jasność jest zmniejszana. W momencie, w którym wynik przekroczy wartości brzegowe piksela, przypisuje się mu p$_{\text{min}}$ lub p$_{\text{max}}$ zgodnie z poniższym wzorem:
\[ p(i) =
  \begin{cases}
    p_{min} & \quad \text{jeżeli } i + k < 0\\
    i + k  & \quad \text{jeżeli } p_{min} \leq i+k \leq p_{max}\\
    p_{max}  & \quad \text{jeżeli } i + k > 0\\
  \end{cases}
\]
gdzie:\\
\textit{p(i)} - wartość piksela po zmianie jasności,\\
\textit{i} - wartość piksela przed zmianą jasności,\\
\textit{p$_{\text{min}}$} - minimalna wartość piksela, p$_{\text{min}}$ = 0,\\
\textit{p$_{\text{max}}$} - maksymalna wartość piksela,\\
\textit{k} - zmiana jasności.\\

\subsubsection{Zmiana kontrastu}
Zmiana kontrastu obrazu polega na zwiększeniu jasności jasnych pikseli przy jednoczesnym zmniejszeniu jasności ciemnych pikseli. Ciemne piksele należą do przedziału $\langle$0,$ \frac{p_{max}}{2}$), zaś jasne do przedziału $\langle$$ \frac{p_{max}}{2}$,$p_{max}$$\rangle$.\\
\\
Wzór wygląda następująco:
\[ p(i) =
  \begin{cases}
    i \div k  & \quad \text{jeżeli } 0 \leq i < \frac{p_{max}}{2}\\
    i \ast k  & \quad \text{jeżeli } \frac{p_{max}}{2} \leq i \leq p_{max}\\
    p_{max}  & \quad \text{jeżeli } i \ast k > p_{max}\\
  \end{cases}
\]
gdzie:\\
\textit{p(i)} - wartość piksela po zmianie kontrastu,\\
\textit{i} - wartość piksela przed zmianą kontrastu,\\
\textit{p$_{\text{max}}$} - maksymalna wartość piksela,\\
\textit{k} - współczynnik zmiany kontrastu.\\

\subsubsection{Wyznaczenie negatywu}
Negatyw jest przedstawieniem pikseli obrazu jako różnicy wartości maksymalnej i obecnej:\\
\\
\[p(i) = p_{max} - i\]
gdzie:\\
\textit{p(i)} - wartość piksela po wyznaczeniu negatywu,\\
\textit{i} - wartość piksela przed zmianą kontrastu,\\
\textit{p$_{\text{max}}$} - maksymalna wartość piksela.\\

\subsection{Podstawowe filtry}
Poniższa sekcja opisuje 2 podstawowe filtry, które podległy analizie - filtr ze średnią arytmetyczną oraz filtr medianowy.

\subsubsection{Filtr ze średnią arytmetyczną}
Filtr ze średnią arytmetyczną jest wykorzystywany w operacjach odszumiania obrazu. Algorytm polega na przypisaniu do nowej wartości pikseli średniej arytmetycznej badanego elementu oraz jego sąsiedztwa s. Sąsiedztwo może przyjmować wartości potęg kolejnych liczb nieparzystych:\\
\[ s = {9,25,49...} \]
\[ p(i) = \displaystyle\sum_{x=-n}^{n}\displaystyle\sum_{y=-n}^{n} i_{x,y} \]
gdzie:\\
\textit{p(i)} - wartość piksela po nałożeniu filtru,\\
\textit{i} - wartość piksela przed nałożeniem filtru,\\
\textit{n} - rozpiętość maski filtru, obliczana ze wzoru:\\
\[ n = \frac{\sqrt{s}-1}{2} \]
\textit{x} - współrzędna x piksela na obrazie\\
\textit{y} - współrzędna y piksela na obrazie\\
\\
Należy pamiętać, że piksele poddawane filtracji nie mogą być elementami brzegowymi obrazu, więc:
\[ i_x + n \leq x_{max} \wedge i_x - n \geq x_{min} \wedge i_y + n \leq y_{max} \wedge i_y - n \geq y_{min} \]
gdzie:\\
\textit{i} - wartość piksela przed zmianą kontrastu,\\
\textit{x} - współrzędna x piksela na obrazie\\
\textit{y} - współrzędna y piksela na obrazie\\
\textit{x$_{\text{max}}$} - maksymalna wartość współrzędnej x na obrazie\\
\textit{x$_{\text{min}}$} - minimalna wartość współrzędnej x na obrazie, x$_{\text{min}}$ = 0\\
\textit{y$_{\text{max}}$} - maksymalna wartość współrzędnej y na obrazie\\
\textit{y$_{\text{min}}$} - minimalna wartość współrzędnej y na obrazie, y$_{\text{min}}$ = 0\\



\subsubsection{Filtr medianowy}
//TODO

\subsection{Modyfikacje obrazu w oparciu o histogram}
//TODO

\subsubsection{Jednostajna wyjściowa gęstość prawdopodobieństwa}
//TODO

\subsubsection{Wyjściowa gęstość prawdopodobieństwa o postaci wykładniczej}
//TODO

\subsubsection{Wyjściowa gęstość prawdopodobieństwa podana wzorem Raleigha}
//TODO

\subsubsection{Wyjściowa gęstość prawdopodobieństwa określona przez potęgę 2/3}
//TODO

\subsubsection{Wyjściowa gęstość prawdopodobienstwa o postaci hiperbolicznej}
//TODO

\subsection{Filtracja liniowa oparta o splot}
//TODO

\subsubsection{Filtr dolnoprzepustowy}
//TODO

\subsubsection{Wyostrzanie krawędzi}
//TODO

\subsubsection{Wydobywanie szczegółów z tła: N, NE, E, SE}
//TODO

\subsubsection{Wydobywanie szczegółów z tła: S, SW, W, NW}
//TODO

\subsubsection{Wydobywanie szczegółów z tła bez zdefiniowanego kierunku (laplasjan)}
//TODO

\subsubsection{Identyfikowanie linii}
//TODO

\subsection{Filtracja nieliniowa}
//TODO

\subsubsection{Operator Robertsa (Wariant I)}
//TODO

\subsubsection{Operator Robertsa (Wariant II)}
//TODO

\subsubsection{Operator Sobela}
//TODO

\subsubsection{Operator Kirsha}
//TODO

\subsubsection{Operator Rosenfelda}
//TODO

\subsubsection{Operator Uolisa}
//TODO

{\color{blue}
We wprowadzeniu należy zaprezentować całą teorię potrzebną do realizacji
zadania (przy czym należy tu ograniczyć się wyłącznie do tego, co było
wykorzystane) tak aby osoba, która nigdy wcześniej nie zetknęła się z tą
tematyką, potrafiła zrozumieć dalszy opis. Część ta powinna wprowadzać
wszystkie wykorzystywane wzory, oznaczenia itp., do których należy się
odwoływać w dalszej części niniejszgo sprawozdania. Zamieszczony tu własny
opis teorii (a nie skopiowany!) należy poprzeć odwołaniami bibliograficznymi
do literatury zamieszczonej na końcu. }

\section{Opis implementacji}
{\color{blue}
Należy tu zamieścić krótki i zwięzły opis zaprojektowanych klas oraz powiązań
między nimi. Powinien się tu również znaleźć diagram UML  (diagram klas)
prezentujący najistotniejsze elementy stworzonej aplikacji. Należy także
podać, w jakim języku programowania została stworzona aplikacja. }

\section{Materiały i metody}
{\color{blue}
W tym miejscu należy opisać, jak przeprowadzone zostały wszystkie badania,
których wyniki i dyskusja zamieszczane są w dalszych sekcjach. Opis ten
powinien być na tyle dokładny, aby osoba czytająca go potrafiła wszystkie
przeprowadzone badania samodzielnie powtórzyć w celu zweryfikowania ich
poprawności (a zatem m.in. należy zamieścić tu opis architektury sieci,
wartości współczynników użytych w kolejnych eksperymentach, sposób
inicjalizacji wag, metodę uczenia itp. oraz informacje o danych, na których
prowadzone były badania). Przy opisie należy odwoływać się i stosować do
opisanych w sekcji drugiej wzorów i oznaczeń, a także w jasny sposób opisać
cel konkretnego testu. Najlepiej byłoby wyraźnie wyszczególnić (ponumerować)
poszczególne eksperymenty tak, aby łatwo było się do nich odwoływać dalej.}

\section{Wyniki}
Sekcja przedstawia efekty przeprowadzonych badań. Przeanalizowane zostały wybrane obrazy 1-, 8- i 24-bitowe.

{\color{blue}
W tej sekcji należy zaprezentować, dla każdego przeprowadzonego eksperymentu,
kompletny zestaw wyników w postaci tabel, wykresów itp. Powinny być one tak
ponazywane, aby było wiadomo, do czego się odnoszą. Wszystkie tabele i wykresy
należy oczywiście opisać (opisać co jest na osiach, w kolumnach itd.) stosując
się do przyjętych wcześniej oznaczeń. Nie należy tu komentować i interpretować
wyników, gdyż miejsce na to jest w kolejnej sekcji. Tu również dobrze jest
wprowadzić oznaczenia (tabel, wykresów) aby móc się do nich odwoływać
poniżej.}

\section{Dyskusja}
{\color{blue}
Sekcja ta powinna zawierać dokładną interpretację uzyskanych wyników
eksperymentów wraz ze szczegółowymi wnioskami z nich płynącymi. Najcenniejsze
są, rzecz jasna, wnioski o charakterze uniwersalnym, które mogą być istotne
przy innych, podobnych zadaniach. Należy również omówić i wyjaśnić wszystkie
napotakane problemy (jeśli takie były). Każdy wniosek powinien mieć poparcie
we wcześniej przeprowadzonych eksperymentach (odwołania do konkretnych
wyników). Jest to jedna z najważniejszych sekcji tego sprawozdania, gdyż
prezentuje poziom zrozumienia badanego problemu.}
\section{Wnioski}
{\color{blue}W tej, przedostatniej, sekcji należy zamieścić podsumowanie
najważniejszych wniosków z sekcji poprzedniej. Najlepiej jest je po prostu
wypunktować. Znów, tak jak poprzednio, najistotniejsze są wnioski o
charakterze uniwersalnym.}


\begin{thebibliography}{0}
\end{thebibliography}
{\color{blue} 
Na końcu należy obowiązkowo podać cytowaną w sprawozdaniu
literaturę, z której grupa korzystała w trakcie prac nad zadaniem (przykład na
końcu szablonu)}
\end{document}
